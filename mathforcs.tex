\documentclass[10pt]{article}
\usepackage{amsmath,amssymb}
\usepackage{amsthm}
%\usepackage{algorithmic} 
\usepackage {ifpdf} 
\pagestyle{headings}
\textheight    8.4in
\textwidth      6.2in
\topmargin        0in
\oddsidemargin  0in
\evensidemargin 0in

\newcounter{set}
\def\theset{\arabic{set}}
\def\theequation{\theset.\arabic{equation}}
\newcommand{\sig}{\sigma}
\newcommand{\ra}{\rightarrow}
\newcommand{\eps}{\epsilon}
\newcommand{\ah}{\alpha}
\newcommand{\lam}{\lambda}
\newcommand{\gam}{\gamma}
\newcommand{\ol}{\overline}
\newcommand{\noi}{\noindent}
\newcommand{\ben}{\begin{enumerate}}
\newcommand{\een}{\end{enumerate}}
\newcommand{\beq}{\begin{quote}}
\newcommand{\enq}{\end{quote}}
\newcommand{\hsone}{\hspace*{1cm}}
\newcommand{\vsone}{\vspace*{1cm}}
\newcommand{\hstwo}{\hspace*{2cm}}
\newcommand{\So}{\\ {\tt Solution:}}

\newtheorem{theorem}{Theorem}[subsection]
\newtheorem{lemma}[theorem]{Lemma}
\newtheorem{proposition}[theorem]{Proposition}
\newtheorem{corollary}[theorem]{Corollary}

\newenvironment{definition}[1][Definition]{\begin{trivlist}
\item[\hskip \labelsep {\bfseries #1}]}{\end{trivlist}}
\newenvironment{example}[1][Example]{\begin{trivlist}
\item[\hskip \labelsep {\bfseries #1}]}{\end{trivlist}}
\newenvironment{remark}[1][Remark]{\begin{trivlist}
\item[\hskip \labelsep {\bfseries #1}]}{\end{trivlist}}

\title{Mathematics for Computer Science - notes} 
\author{Angjoo Kanazawa}
\date{\today}
\setcounter{set}{1}


\begin{document}

\maketitle \thispagestyle{empty}
\setcounter{section}{7}
\section{Chapter 8 Communication Networks}
\label{sec:chapt-8-comm}

\subsection{Complete Binary Tree}
\label{sec:complete-binary-tree}

Definitions/formulas:
\begin{itemize}
\item \textbf{Switches} direct packets through the network (circle nodes)
\item \textbf{Latency} is the time required for a packet to travel from an input to an output.
\item The \textbf{diameter} of a network is the number of switches on the shortest path between
the input and output that are farthest apart. Approxmiates the
worst-case latency.
\item The diameter of a complete binary tree with N in/outputs is
  $2logN+1$. Not bad, even with $2^{10}$ terminals, latency is
  $2log(2^{10})+1 = 21$
\item Total number of switches in a complete binary tree is $2N-1$.
\item A \textbf{permutation} is a bijective function $\pi: {0, 1,\dots, N-1} \ra
  {0, 1, \dots, N-1}$
\item $\forall \pi \exists$ \textbf{permutation routing problem},
  where the challenge is to direct a packet starting at input $i$ to
  output $\pi(i)$
  \begin{itemize}
  \item Solution: specification fo the path taken for all $N$
    packets. Where the path of packet $i$ to output $\pi(i)$ is
    denoted $P_{i, \pi(i)}$.
  \item The \textbf{congestion} of a set of paths $P_{0,\pi(0)},
    \dots, P_{N-1, \pi(N-1)} $, is the largest number of paths that pass
    through a single switch. i.e for $\pi(i) = i$, the congestion is 1,
    for $\pi(i) = (N-1)-i$, the congestion is 4. Lower the congestion,
    the better the set of paths.
  \item The \textbf{Max congestion} of a network is ``maximum over all
    permutations $\pi$ of the minimum over all paths' congestion...
    $\arg\max_{\pi}\arg\min_{\text{all possible paths}}Cong(P_{i, \pi(i)})$ where $Cong$ is the congestion of a given path?
  \item The max congestion
    of a complete binary tree is $N$ with $\pi(i) = (N-1)-i$, because
    with that $\pi$, every packet needs to go through the root. You can't do better and its
    horrible.
  \item over all permutation.. = $N!$ permutations for $N$ in/out
    networks. For each of those permutations, there are paths that
    takes $i$ to $\pi(i)$ = a lot of paths, but chose the BEST
    congestion. And the max of those out of $N!$ perm is the max congestion..
  \end{itemize}


\end{itemize}

Goals:
\begin{itemize} 
\item larger the switches $\ra$ smaller diameter. Most nodes in a binary
  tree takes 3 in/out edges => 3x3 switches
\item Want: how to get NxN monster switch using 3x3 simple switches
\end{itemize}

\subsection{2-D Array/Grid/Crossbar}
\label{sec:2-d-array}
\begin{itemize}
\item Diameter of an array with $N$ in/outputs is $2N-1$.
\item Each switch takes two in/out edges, so switch size is $2x2$.
\item \# of switches is the number of elem in $NxN$ array = $N^2$
\end{itemize}


\begin{theorem}
The (max) congestion of an N-input array is 2
\end{theorem}
 \begin{proof} 
It's at most 2: Let $\pi$ be any permutation. Let $P_{i, \pi(i)}$ to be the path fro
input $i$ rightward to col $j$, downward to output $\pi(i)$ (can go
other ways but this is the best path), so the $(i, j)$th switch transmits at most 2 packets. The one coming from
left and the one coming from the top..?\\
It's at least 2: With $\pi(0) = 0$ and $\pi(N-1) = N-1$, the packet in the lower left
corner must pass two packets. (It's at least so just show it exists).
  \end{proof}

\subsection{Butterfly}
\label{sec:butterfly}

\begin{itemize}
\item All terminals and switches are in $N$ rows, where inputs,
  ordered, are the first col, output, ordered, are the last col. Now
  label the rows in binary, so row $i$ has binary number
  $b_1b_2\cdots b_{logN}$ that represents $i$.
\item $\exists$ $log(N) + 1$ levels of switches (nodes), numbered from
  $0$ to $logN$. Each level is a column of $N$ switches. So every
  switch has a unique sequence $(b_1, b_2, \dots, b_{logN}, l)$, where
  $b_1b_2\cdots b_{logN}$ is the switch's row label and $l$ is the
  level of the switch ($0\ra logN$).
\item So... $Nby1\ra Nbylog(N)+1 \ra Nby1$
\item Connection: there are directed edges from $(b_1, b_2, \dots,
  b_{logN}, l)$ to two switches in the next level, one in the same
  row, one in the row with label $\bar{l+1}$ (inverting bit $l+1$).
\item Recursive structure. A butterfly of size $2N$ is made up of two
 $N$  butterflies plus one more level of switches. 
\end{itemize} There is only one path from an input to an output, where the path is by correcting each
successive bit with the bit of the output.\\

\begin{corollary}The congestion of the butterfly network is exactly
  $\sqrt{N}$ when N is an even power of 2:
\end{corollary}
\begin{proof}
Let $B_n$ denote the butterfly network with $N=2^n$ inputs and $N$ outputs.

For $B_n$, there is a unique path from each input to each output,
so congestion is the max number of transmission for a vertex.
(The number of total vertices is $N(log(N)+1)$, or $2^n(n+1)$... useless)

For every vertex $v$ at level $i$, there's a path from exactly $2^i$
input vertices to $v$ and exactly $2^{n-i}$ output vertices (where $n$
is the power to 2).


Since there is a unique path from each input to output,  the number of
messages that passes through a vertex is at most the minimum of number
of input or output vertices it has a path from/to. 
 
So congestion must be worst at the center level of the
network. i.e for $n=3$, level 0 vertices have a path from 1 input
vertex and have a path to 8 output vertices, level 1 has 2-from and
4-to, level 2 has 4-from and 2-to, and the last level has 8-from and
1-to. Now there are $n+1$ levels (or $log(N)+1$). So with even $n$,
there exists a center level with vertices that have a path from exactly $2^{(n/2)}$
input vertices, and a path to exactly $2^{n/2}$ output vertices.

Which means that congestion of vertices at the middle level is at most
$\sqrt{N} = 2^{n/2}$. 

... not clear which one is at least and at most.. It's like.. at most
and at least because in the middle its $\sqrt{N}$ from and
$\sqrt{N}$..
 \end{proof}

\subsection{Benes Network}
A back to back butterfly.
\begin{itemize}
\item Diameter: $2logN$, switch size 2 by 2, \# of switches $2NlogN$,
  congestion 1!!
\end{itemize}

\begin{theorem}
  The congestion of the N-input Benes network is 1 , where $N=2^a$ for
  some $a\le 1$.
\end{theorem}
\begin{proof}
Inductive:  Let $P(a)$ be the proposition that congestion of the size $2^a$
  Benes network is 1.

Base case: WTS: congestion of $N=2^1 = 2$ Benes network is 1. For 2
input network, there are $2!=2$ permutation, for identity, we can just
take the straight path, for the other one, they'll just take the
diagonal. So congestion is 1.

Inductive arg: WTS: $P(a) \ra P(a+1)$ where $a\le 1$. Key insight: a
2-coloring of the constraints graph corresponds to a solution to the
routing problem. If we show that the graph is 2-colorable, that means
we can solve the routing problem with such constraints.

Here's a theorem:
\begin{theorem}
  If the graphs $G_1(V, E_1)$ and $G_2 = (V, E_2)$ both have max
  degree 1, then th egraph $G = (V, E_1\cup E_2)$ is \emph{2-colorable}.
\end{theorem}

Using that idea, let $\pi$ be any permutation of ${0, 1, \dots,
  2N-1}$. Define two graphs with those vertices, but one where with edge $u-v$ with $|u-v|=N$,
and one with edge $u-v$ with $|\pi(u)-\pi(v)} = N$. By the above
theorem, the graph $G=(V, E_1\cup E_2)$ is 2-colorable. Route one
color with the upper subnetwork and one color with lower
subnetwork. By induction hypothesis $P(a)$, the subneworks can each
route any permutation with congestion 1, and so we're done. (We only needed to show that
the first and the last levels have cogestion 1)
\end{proof}
\end{document}



%% example of proof in latex
%  \begin{proof}[$A \cup (A \cap A^c) = A\cap ( A\cup A^c)$]
%     \begin{align*}
% A \cup (A \cap A^c) &= (A \cup A) \cap (A\cup A^c)\text{ by
%   distributive law, Theorem 1.5.c}\\
% &= A\cap ( A\cup A^c) \text{ by idenpote,
%   Theorem 1.5.g}
%     \end{align*}
%   \end{proof}
