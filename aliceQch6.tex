\documentclass[10pt]{article}
\usepackage{amsmath} 
\usepackage{ifpdf} 
\pagestyle{headings}
\textheight    8.4in
\textwidth      6.2in
\topmargin        0in
\oddsidemargin  0in
\evensidemargin 0in

\newcounter{set}
\def\theset{\arabic{set}}
\def\theequation{\theset.\arabic{equation}}

\title{Probability and Computing Chapter 6 Notes and Questions} 
 \markboth{Chp6}{}
\author{Angjoo Kanazawa}

\date{\today}
\setcounter{set}{1}


\begin{document}
\maketitle \thispagestyle{empty}

\section{Chapter 6 The Probabilistic Method}
\label{sec:chap6}


\subsection{Notes}
\begin{itemize}
\item To prove teh existence of an object with specific properties,
  construct an approximate probability space of objects $\mathit{S}$,
  and show that the probability that an object (in $\mathit{S}$ with
  the specific proprties) is selected is $> 0$. Strictly.
\item 
\end{itemize}

\subsection{Exercises}

\textbf{6.1} 
\label{Q6.1}
Consider an instance of SAT with $m$ clauses, where every clause has
exactly $k$ literals.
\begin{enumerate}
\item[(a)]  Give a Las Vegas algorithm that finds an
assignment that satisfies at least $m(1-2^{-k})$ clauses, anayze its
expected running time:\\

The first part is straight from the book (6.2.2). Note:
$$P(\text{a clause is satisfied}) = 1 - P(\text{all literals are
  false}) = 1 - 2^{-k}$$
Let $i= 1\dots m$, $X_i = 1$ if $i$th clause is satisfied, $X_i=0$
otherwise. 
Let $X = \sum_{i}^m X_i$, the total number of satisfied clauses.
So we get $$E(X) = \sum_i^m X_iP(X_i=1) = m(1-2^{-k}) = \mu$$

Now given this, the LV algorithm goes like this:

Repeat until $X \ge \mu$:
\begin{itemize}
\item assign values to all boolean variables independently and uniformly.
\item Check the value of $X$.
\end{itemize}

What is the expected number of runs until LV finishes? We're done
when $X \ge \mu$. Note that $X$
has a binomial distribution*. Let $Y$ be the number of runs needed for
the LV to terminate, and $Y_i$ be a random variable indicating if the
algorithm terminated at $i$th run or not with probability $\hat{p} =
P(Y_i = 1)$. So $Y$ has a geometric distribution and $E(Y)$, what we
want, is
\begin{equation}
  \label{eq:1}
E(Y) = \sum_{I=1}^\infty i\hat{p}(1-\hat{p})^{i-1}
\end{equation}


So for each run, what is $\hat{p} = P(Y_i = 1) = P(X\ge \mu)$? Using $p
= P(\text{a clause is satisfied}) = 1 - 2^{-k}$:
\begin{align*}
  P(X\ge \mu) &= 1 - P(X < \mu)\\
&= 1 - [ P(X=0) + P(X=1) + \cdots + P(X = \mu -1)]\\
&= 1 - [(1-p)^m + {m \choose 1}p(1-p)^{m-1} + \cdots + {m
 \choose \mu - 1} p^{\mu-1}(1-p)^{m -(\mu -1)}]\\
&= 1 - \sum_{i=0}^{\mu-1}{m \choose i}p^i(1-p)^{m-i}\\
\end{align*}

Now recall $\sum_i^m{m \choose i} = 2^m$, so $\sum_i^{\mu -1} {m
  \choose i} \le \frac{2^m}{2} = 2^{m-1}$. 

Also, notice that here $p^i(1-p)^{m-i} = (1-2^{-k})^i(1-2^{-k})^{m-i}
= (1-2^{-k})^m$.

So we can continue the above inequalitywith:
\begin{align*}
  P(X\ge \mu) &= 1 - \sum_{i=0}^{\mu-1}{m \choose i}p^i(1-p)^{m-i}\\
  &\ge 1 -[ 2^{m-1} (1-2^{-k})^m] = 1-2^{-k}
\end{align*}
As an example with $k=1$, this $P(X\ge \mu)=\hat{p}$ is just $1/2$.
Using that \ref{eq:1}, we get $E[Y] = ...$. The
algorithm is.. very efficient.

...\\Really..? * is where I'm not sure. Perhaps this is just too
much. Can one always say $P(X\ge \mu) \ge 1/2$?? I wasn't sure.

\item[(b)] Give a derandomization of the randomized algorithm using
  the method of conditional expectations:\\
This I also just followed the book. Maybe too closely. 

We know setting variables independently and uniformly gives us $E(X)
\ge m(1-2^{-k})$. Now set the boolean variables $x_1, x_2, \dots$  up
to $r$ deterministically one at a time.

Consider the expected total \# of satisfied clauses if the remaining
boolean variables are selected independently and uniformly. Write this
as $E(X|x_1, x_2, \dots, x_r)$. We want a away o set the next variable
s.t.
\begin{equation}
  \label{eq:2}E(X|x_1, \dots, x_r) \le E(X|x_1, \dots, x_r, x_{r+1})  
\end{equation}
Inductively, the base case is $E(X|x_1) = E(X)$. Now, consider setting
$x_{r+1}$ randomly to true or false. Each has probability $1/2$. So
$E(X|x_1,\dots,x_r) = \frac{1}{2}E(X|x_1,\dots,x_{r+1} = 1) +
\frac{1}{2}E(X|x_1,\dots, x_{r+1} = 0)$
From this we can deduce $$\max(E(X|x_1,\dots,x_r,x_{r+1} = 1),
E(X|x_1,\dots,x_r,x_{r+1} = 0)) \ge E(X|x_1,\dots, x_r)$$ 

So we just have to chose the assignment that increases the
conditional expectation the most.. we only have two options $x_{r+1}$
is T or F, so look at clauses that contain the $x_{r+1}$ variable
twice and see how the expectation changes based on the assignment and
take the better one? Something like that.

\end{enumerate}






\end{document}
